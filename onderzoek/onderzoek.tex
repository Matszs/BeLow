\documentclass{below-ext}

\title{A standard for Home monitoring}

\numberofauthors{5}
\author{
  \vspace{-1.5em} 
  \alignauthor{
  	\textbf{Patrick Hendriks}\\
  	\email{patrick.hendriks@hva.nl}
  }
  \vfil
  \alignauthor{
  	\textbf{Mats Otten}\\
  	\email{mats.otten@hva.nl}
  }
  \vfil
  \alignauthor{
  	\textbf{Suzanne Peerdeman}\\
  	\email{suzanne.peerdeman@hva.nl}
  }
  \vfil
  \alignauthor{
  	\textbf{Hogeschool van Amsterdam}\\
  	\affaddr{Wibautstraat 2-4}\\
  	\affaddr{1091 GM, Amsterdam}\\
  	\email{suzanne.peerdeman@hva.nl}
  }
  \vfil
  \alignauthor{
  	\textbf{Glimworm IT BV}\\
  	\affaddr{Kattenburgerstraat 5}\\
  	\affaddr{1018 JA Amsterdam}\\
  	\email{suzanne.peerdeman@hva.nl}
  }
}


% Paper metadata (use plain text, for PDF inclusion and later re-using, if desired)
\def\plaintitle{A standard for Home monitoring}
\def\plainauthor{Patrick Hendriks}
\def\plainkeywords{Healthcare, Technology, Elderly}
\def\plaingeneralterms{Research}

\hypersetup{
  % Your metadata go here
  pdftitle={\plaintitle},
  pdfauthor={\plainauthor},  
  pdfkeywords={\plainkeywords},
  pdfsubject={\plaingeneralterms},
  % Quick access to color overriding:
  %citecolor=black,
  %linkcolor=black,
  %menucolor=black,
  %urlcolor=black,
}

\usepackage{graphicx}   % for EPS use the graphics package instead
\usepackage{balance}    % useful for balancing the last columns
\usepackage{bibspacing} % save vertical space in references
\usepackage{ragged2e} % alignment


\begin{document}

\maketitle

\begin{abstract}
Lorem ipsum dolor sit amet, consectetur adipiscing elit. Vivamus interdum vehicula magna ut imperdiet. Ut ac sagittis quam. Pellentesque accumsan aliquam lacinia. Maecenas vitae aliquet mauris, ut lobortis est. Morbi pulvinar ex ipsum, id semper enim egestas eget. Vestibulum tortor justo, maximus ut nulla in, imperdiet sagittis neque. Nulla facilisi. Curabitur pretium mollis turpis, eget congue quam dapibus vel. Curabitur molestie mauris nisi, vulputate consectetur lacus aliquam ut. Duis lorem massa, porttitor posuere faucibus vel, blandit vel libero. Curabitur pretium, ipsum quis tincidunt aliquet, ante mauris tincidunt mi, eget pellentesque odio ipsum quis odio. Fusce ultrices lacus sed velit faucibus, vel eleifend neque vestibulum. Donec ultricies aliquam tristique. Praesent placerat vitae felis ut maximus. Praesent sit amet tincidunt felis. Mauris sollicitudin sagittis neque.
\end{abstract}

%\keywords{\plainkeywords}
%\textcolor{red}{Mandatory section to be included in your final version.}

%\category{H.5.m}{Information interfaces and presentation (e.g., HCI)}{Miscellaneous}. 
%See \cite{ACMCCS} 
%See: \url{http://www.acm.org/about/class/1998/} 
%for help using the ACM Classification system.
%\textcolor{red}{Mandatory section to be included in your final version.}

%\terms{\plaingeneralterms}
%\textcolor{red}{Optional section to be included in your final version.}


% =============================================================================
\section{Introduction}

% =============================================================================
\justifying
For the past several years, the changing demographic in the Netherlands has raised many concerns. The growing number of elderly people living alone and the question of how to care for the ageing population have caused a surge of technological developments aimed specifically at this targed audience. However, even if a demand for ambient intelligence based systems such as these, the market does not quite seem to take off in the way that most developers hope for. Conduction of several focus groups among elderly people living independently has proven that the demand exists, yet is often not satisfied by what is currently on offer. Various systems target the caregivers of those that will actually be using the product, rather than the users themselves. It has become apparent that a large portion of the elderly population wishes to be respected in their autonomy, and take matters into their own hands.

The products available are also specifically targeted at an elderly audience, even though they would be able to cater a much more diverse group of users. Therefore, the goal of BeLow is to lay the foundation for a portable living lab; a standard kit that can be used by anyone, even those with minimal technical experience, and that can be expanded upon by those that wish to do so.

A major part of this research was performed within Stedeborgh; a living community by the elderly for the elderly. Stedeborgh is a largely autonomous community, with residents aged 67 and up. For this paper several tests, focus groups and interviews were conducted among the residents, and their sentiments, routines and opinions have played a big part in the decisions that were made. Aside from Stedeborgh, this research was performed in collaboration with Amsterdam's Digital Life Centre's project BRAVO (led by Saskia Robben, 1 Apr 2016 - 1 Apr 2018) and Glimworm Information Technology (Adrian Blackwood). Collectively, the question we ask ourselves is: "How can we develop an accessible standard with which to add a layer of intelligence to the living environment?" 

In this paper the focus will be on answering the aforementioned question, as well as answering several other questions:
\begin{itemize}
\item What monitoring systems are currently on the market, and what knowledge is available?
\item  What are the wishes of the target audience when it comes to improvement of living quality using technology?
\item How can the needs of the target audience be satisfied using technology.
\item How does one become a standard?
\end{itemize}

This paper will begin with a report on literature and field studies performed to answer the first two of these questions. The second half will consist of reports on experiments and tests performed using prototypes. Finally, we will look to the future and, based on interviews with experts on the subject, discuss how technology may develop in the years to come and how this can be of use to society as a whole.

% =============================================================================
%\section{Copyright}
% =============================================================================
%Copyright tekst

% =============================================================================


%\section{Acknowledgements}
%We thank all DUX 2003 publications support and staff who wrote this document originally and allowed us to modify it for this conference.
%This template was based on Manas Tungare's \texttt{chi.cls}, and rewritten by Luis A. Leiva.

\section{References format}
References must be the same font size as other body text.
% REFERENCES FORMAT
% References must be the same font size as other body text.

\balance
\bibliographystyle{acm-sigchi}
\bibliography{refs}

\end{document}
