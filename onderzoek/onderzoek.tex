\documentclass{below-ext}

\title{A standard for Home monitoring}

\numberofauthors{5}
\author{
  \vspace{-1.5em} 
  \alignauthor{
  	\textbf{Patrick Hendriks}\\
  	\email{patrick.hendriks@hva.nl}
  }
  \vfil
  \alignauthor{
  	\textbf{Mats Otten}\\
  	\email{mats.otten@hva.nl}
  }
  \vfil
  \alignauthor{
  	\textbf{Suzanne Peerdeman}\\
  	\email{suzanne.peerdeman@hva.nl}
  }
  \vfil
  \alignauthor{
  	\textbf{Hogeschool van Amsterdam}\\
  	\affaddr{Wibautstraat 2-4}\\
  	\affaddr{1091 GM, Amsterdam}\\
  	\email{suzanne.peerdeman@hva.nl}
  }
  \vfil
  \alignauthor{
  	\textbf{Glimworm IT BV}\\
  	\affaddr{Kattenburgerstraat 5}\\
  	\affaddr{1018 JA Amsterdam}\\
  	\email{suzanne.peerdeman@hva.nl}
  }
}


% Paper metadata (use plain text, for PDF inclusion and later re-using, if desired)
\def\plaintitle{A standard for Home monitoring}
\def\plainauthor{Patrick Hendriks}
\def\plainkeywords{Healthcare, Technology, Elderly}
\def\plaingeneralterms{Research}

\hypersetup{
  % Your metadata go here
  pdftitle={\plaintitle},
  pdfauthor={\plainauthor},  
  pdfkeywords={\plainkeywords},
  pdfsubject={\plaingeneralterms},
  % Quick access to color overriding:
  %citecolor=black,
  %linkcolor=black,
  %menucolor=black,
  %urlcolor=black,
}

\usepackage{graphicx}   % for EPS use the graphics package instead
\usepackage{balance}    % useful for balancing the last columns
\usepackage{bibspacing} % save vertical space in references
\usepackage{ragged2e} % alignment


\begin{document}

\maketitle

\begin{abstract}
\ldots
\end{abstract}

%\keywords{\plainkeywords}
%\textcolor{red}{Mandatory section to be included in your final version.}

%\category{H.5.m}{Information interfaces and presentation (e.g., HCI)}{Miscellaneous}. 
%See \cite{ACMCCS} 
%See: \url{http://www.acm.org/about/class/1998/} 
%for help using the ACM Classification system.
%\textcolor{red}{Mandatory section to be included in your final version.}

%\terms{\plaingeneralterms}
%\textcolor{red}{Optional section to be included in your final version.}


% =============================================================================
\section{Introduction}

% =============================================================================
\justifying
For the past several years, the changing demographic in the Netherlands has raised many concerns. The growing number of elderly people living alone and the question of how to care for the ageing population have caused a surge of technological developments aimed specifically at this targed audience. However, even if a demand for ambient intelligence based systems such as these, the market does not quite seem to take off in the way that most developers hope for. Conduction of several focus groups among elderly people living independently has proven that the demand exists, yet is often not satisfied by what is currently on offer. Various systems target the caregivers of those that will actually be using the product, rather than the users themselves. It has become apparent that a large portion of the elderly population wishes to be respected in their autonomy, and take matters into their own hands.

The products available

% =============================================================================
%\section{Copyright}
% =============================================================================
%Copyright tekst

% =============================================================================


%\section{Acknowledgements}
%We thank all DUX 2003 publications support and staff who wrote this document originally and allowed us to modify it for this conference.
%This template was based on Manas Tungare's \texttt{chi.cls}, and rewritten by Luis A. Leiva.

\section{References format}
References must be the same font size as other body text.
% REFERENCES FORMAT
% References must be the same font size as other body text.

\balance
\bibliographystyle{acm-sigchi}
\bibliography{refs}

\end{document}
